
\documentclass{article}

\newcommand{\steem}{\texttt{STEEM}}
\newcommand{\mytoken}{\texttt{MYTOKEN}}

\begin{document}

Creators: @theoretical; @youkaicountry
Contributors: @ned; @sneak; @vandeberg; @valzav; @picokernel
Copyright (c) Steemit, Inc. 2017

\section{Introduction}

\section{Setup}

\subsection{Basic definitions}

In this article, we'll let $s$ represent a quantity of STEEM, let $t$
represent a quantity of some token (SMT), and let $p$ represent a price,
such that $pt$ is STEEM-valued (i.e. if MYTOKEN is trading at
$p = 0.05$ STEEM / MYTOKEN then $t = 120$ MYTOKEN has a value of
$pt = (0.05\ \mbox{STEEM / MYTOKEN}) \cdot (120\ \mbox{MYTOKEN}) = 6\ \mbox{STEEM}$.

Suppose we have a market maker (or any economic agent) with a two-asset
``portfolio'' (inventory) of $s$ STEEM and $t$ tokens.  If the price of
tokens is $t$, then we may measure of the value of this portfolio, in
units of STEEM, as $v(p, s, t) = s + pt$.

One common portfolio management policy is to require that STEEM
should be some constant fraction $r$ of the portfolio, i.e.
$s = r v(p, s, t)$ where $0 < r < 1$.  We call this policy the
\textit{constant portfolio ratio} or CPR policy, and the equation
$s = r v(p, s, t)$ is the \textit{CPR invariant}.

A different portfolio management policy, discussed by the Bancor
whitepaper, is called CRR or \textit{constant reserve ratio}.  To discuss CRR,
let us notate the total number of tokens in existence as $T$.  The
\textit{CRR invariant} is then defined as $s = r v(p, 0, T-t)$.

\subsection{Note on conventions}

We must discuss where our convention varies from the Bancor whitepaper.
At some times, when some user Alice interacts with the market maker,
Alice will remove some tokens from her balance to get STEEM from
the market maker's balance.  On the other hand, Bob may add some tokens
to his balance in exchange for sending STEEM to the market maker's balance.

Bancor takes the convention that in this example, the market maker \textit{destroys}
tokens in its interaction with Alice, and \textit{creates} tokens in its interaction
with Bob.  The Bancor convention suggests the market maker is not
an ordinary actor, but needs system-level ``special powers'' -- specifically,
the privilege to operate the token printing press -- in order to function.

In this paper, we adopt the convention that the tokens sent by Alice to the
market maker are not destroyed, but are instead added to the inventory (balance)
of the market maker.  Likewise, the tokens sent to Bob by the market maker
are not created out of thin air; they already exist and are merely
transferred from the inventory of the market maker to Bob.  Thus, we show
that the market maker is essentially an ordinary economic agent acting according
to a deterministic algorithm -- it doesn't actually need ``special powers''!

\section{Finite trades}

\subsection{Basic definitions}

A \textit{trade} is a change in the market maker's balance from
$(s, t) \to (s+\Delta s, t+\Delta t)$.  The \textit{price} at which the
trade occurs is defined as $p = {- \Delta s \over \Delta t}$.  We restrict
ourselves to \textit{well-formed trades} where either $\Delta s = \Delta t = 0$,
or $\Delta s$ and $\Delta t$ are both nonzero and have opposite sign.

Theorem:  A trade at price $p$ conserves value at price $p$.  More rigorously,
if $\Delta s, \Delta t$ represent a trade at price $p$, then
$v(p, s, t) = v(p, s + \Delta s, t + \Delta t)$.

Let us more rigorously define the market maker's \textit{state} as a tuple
$M = (s, t, T, r)$.  Given some price $p$, we may define the
\textit{restoring trade} at $p$ (also called a \textit{relaxing trade} or
a \textit{relaxation}) to be a trade which occurs at price $p$ and results in a
state that satisfies the CRR invariant.

\subsection{Computing the restoring trade}

The restoring trade consists of functions $\Delta s(M, p)$ and $\Delta t(M, p)$.
We may actually compute these functions from the definition of price and the
CRR invariant:

\begin{eqnarray*}
\Delta s & = & -p \Delta t \\
s + \Delta s & = & r v(p, 0, T-(t+\Delta t)) \\
\Rightarrow s - p \Delta t & = & r v(p, 0, T-t-\Delta t) \\
   & = & r p(T-t-\Delta t) \\
   & = & rpT - rpt - rp \Delta t \\
\Rightarrow r p \Delta t - p \Delta t & = & r p (T-t) - s \\
\Rightarrow \Delta t & = & {r p (T - t) - s \over rp - p } \\
                     & = & \left( {1 \over 1-r} \right) \left( {s \over p} - r (T-t) \right) \\
\Rightarrow \Delta s & = & - p \Delta t \\
   & = & \left( {1 \over 1-r} \right) \left( r p (T - t) - s \right)
\end{eqnarray*}

\subsection{Computing the equilibrium price}

Given a state $M$, there exists some price $p_{eq}(M)$ for which the restoring
trade is zero; call this price the \textit{equilibrium price}.  We may compute
the equilibrium price by setting $\Delta s = 0$:

\begin{eqnarray*}
\Delta s & = & 0 \\
& = & \left( {1 \over 1-r} \right) \left( r p_{eq} (T - t) - s \right) \\
\Rightarrow r p_{eq} (T - t) - s & = & 0 \\
\Rightarrow r p_{eq} (T - t) & = & s \\
\Rightarrow p_{eq} & = & {s \over r (T-t)}
\end{eqnarray*}

Theorem:  Relaxation is idempotent.  That is, after relaxing at price $p$,
the equilibrium price of the resulting state is $p$, and a second relaxation
at price $p$ will be a zero trade.

\subsection{Example}

Example:  Suppose $M = (1200, 3600, 12000, 0.25)$ and $p = 0.5$.  Then of the
$T = 12000$ TOKEN in existence, $t = 3600$ TOKEN is held by the MM, so
$T-t = 12000 - 3600 = 8400$ TOKEN are ``circulating'' (i.e. exist in balances
outside the MM).  These circulating tokens are worth $p(T-t) = 4200$ STEEM
total, so they ``should be'' backed by a target reserve level of
$rp(T-t) = 4200 * 0.25 = 1050$ STEEM.

In this example, there is ``too much'' STEEM in the reserve, so relaxation
will buy tokens in the market.  This sale will cause two effects:   It will
decrease the reserve STEEM, and also decrease circulating tokens.  The decrease
in circulating tokens, in turn, causes the target reserve level to decline.  For
every 1 STEEM used to buy tokens, the target reserve level declines by $r$ STEEM;
since $r < 1$ eventually the declining reserve will ``catch up'' to its more slowly
declining target level.

The above algebra shows that we will catch up at
$\Delta s = \left( {1 \over 1-r} \right) \left( r p (T - t) - s \right)$ and
$\Delta t = \left( {1 \over 1-r} \right) \left( {s \over p} - r (T - t) \right)$.
Running the calculations with the numbers defined in this example gives
$\Delta s = -200$ STEEM and $\Delta t = 400$ TOKEN.

Let's check that these computed values $\Delta s = -200, \Delta t = 400$
(a) represent a trade with price $0.5$, and (b) that the CRR invariant holds
for the new state $M_{new} = (s+\Delta s, t+\Delta t, T, r)$.  Calculating
$p = {- \Delta s \over \Delta t}$ we indeed get $p = 0.5$.  After this trade executes,
the market maker has $s_{new} = s + \Delta s = 1200 - 200 = 1000$ STEEM, and
$t_{new} = t + \Delta t = 3600 + 400 = 4000$ tokens.

To check condition (b), that the CRR invariant holds, we effectively repeat the
analysis in the initial paragraph of this example with the new numbers.  We know
$M_{new} = (1000, 4000, 12000, 0.25)$ and $p = 0.5$.  Then of the $T = 12000$
TOKEN in existence, $t_{new} = 4000$ TOKEN is now held by the MM, so
$T-t_{new} = 12000 - 4000 = 8000$ TOKEN are now circulating.  These circulating
tokens are worth $p(T-t_{new}) = 4000$ STEEM total, so they ``should be'' backed
by a target reserve level of $rp(T-t) = 4000 * 0.25 = 1000$ STEEM.  Since the
target reserve level indeed exactly matches the actual reserve level of
$s_{new} = 1000$ STEEM, we conclude that the CRR invariant is satisfied after
this relaxing trade.

\section{Infinitesimal trades}

This section is fairly technical; the reader will need a good grasp of
calculus and differential equations to follow the results.

\subsection{Setting up the problem}

Suppose we satisfy the invariant condition at some price
$p = p_eq$; by the CRR invariant $s = r v( p, 0, T-t) = r p (T-t)$.
Suppose the price then increases to $p + \Delta p$ and a relaxing trade
$\Delta s, \Delta t$ occurs at this new price.

In this section we consider the limiting situation where $\Delta p$ is
infintesimally small, so we will use Leibniz notation ($dp$ for a small change
in $p$, $ds$ for a small change in $s$, $dt$ for a small change in $t$).

\subsection{Solving the DE's}

By applying the substitution $p \gets p + dp$ to the expression for $\Delta s$
computed in the previous section, we obtain an expression which simplifies
to a separable DE which can be solved:

\begin{eqnarray*}
ds & = & {1 \over 1-r} \left( r (p + dp) (T - t) - s \right) \\
   & = & {1 \over 1-r} \left( r p (T - t) + r dp (T - t) - r p (T - t) \right) \\
   & = & {1 \over 1-r} r (T - t) dp \\
   & = & {1 \over 1-r} \left( {s \over p} \right) dp \\
\Rightarrow {1 \over p} dp & = & (1-r) \left( {1 \over s} ds \right) \\
\Rightarrow \int {1 \over p} dp & = & (1-r) \int {1 \over s} ds \\
\Rightarrow \ln(p) & = & (1-r) \ln(s) + C_0 \\
\Rightarrow p & = & k_0 s^{1-r}
\end{eqnarray*}

Similarly for $t$, we can start from $dt = -ds / p$ and again obtain and solve
a separable DE:

\begin{eqnarray*}
dt & = & -ds / p \\
   & = & -{r \over 1-r} (T - t) dp / p \\
\Rightarrow {1 \over T-t} dt & = & -{r \over 1-r} \left( {1 \over p} \right) dp \\
\Rightarrow {1 \over p} dp & = & {1-r \over r} \left( {1 \over t-T} \right) dt \\
\Rightarrow \int {1 \over p} dp & = & {1-r \over r} \int {1 \over t-T} dt \\
\Rightarrow \ln(p) & = & {1-r \over r} \ln | t-T | + C_1 \\
\Rightarrow p & = & k_1 (T-t)^{1-r \over r}
\end{eqnarray*}

\section{Qualitative discussion}

In a CRR market maker, where does the ``backing'' for newly emitted tokens come from?

One option is to lower the reserve ratio $r$.  This option results in no immediate market
activity, but will weaken the response of the market maker to any future price changes.
This is called the ``pay later'' option.

Another option is to change the dynamical system's initial conditions, i.e. edit the
constants of integration.  This option will cause the equilibrium price $p_{eq}$ to drop,
meaning the market maker will more aggressively sell tokens to replenish the reserve.
If order books are deep compared to the amount of emission, and there are adequate buyers
for the tokens, then the sales will be able to replenish the reserve and keep the equilibrium
price near its old value; the deep order books provide resistance to the price change being
driven by the market maker.  If order books are thin compared to the amount of emission,
and there are few/no buyers for the tokens, then the equilibrium price will fall, breaking
through the thin orders and lowering the market price.  Even though few/no many tokens were
sold, so even though the \textit{absolute} amount of STEEM in the reserve is still
nearly/exactly the same as before, the reserve's value \textit{relative} to the now-lower
market cap of the token has increased to the reserve ratio.  This option is the ``pay now''
option.




\section{FAQ}

FAQ

Q:  Where are the pretty pictures and graphs?

A:  They're being worked on right now, and will exist in a future version of this paper.

Q:  What is the relevance of constant portfolio ratio policy?

A:  It may become a supported market maker policy in the future.

Q:  Can the reserve ratio go over 100 percent?

A:  No.

Q:  Can the reserve ratio be exactly 100 percent?

A:  Not with the system described in this paper.  It might be possible to code as a special case.

Q:  In a CRR market maker, where does the ``backing'' for newly emitted tokens come from?

A:  As blockchain designers, we have two options for sourcing the ``backing''.  One option
is to lower the reserve ratio $r$.  This option results in no immediate market
activity, but will weaken the response of the market maker to any future price changes.
This is called the ``pay later'' option.

Another option is to change the dynamical system's initial conditions, i.e. edit the
constants of integration.  This option will cause the equilibrium price $p_{eq}$ to drop,
meaning the market maker will more aggressively sell tokens to replenish the reserve.
If order books are deep compared to the amount of emission, and there are adequate buyers
for the tokens, then the sales will be able to replenish the reserve to its target level while
keeping the equilibrium price near its old value.  The deep order books provide resistance to
the price change being driven by the market maker.

If order books are thin compared to the amount of emission, and there are few/no buyers for the
tokens, then the equilibrium price will fall, breaking through the thin orders and lowering
the market price.  Even though few/no many tokens were sold, so even though the
\textit{absolute} amount of STEEM in the reserve is still nearly/exactly the same as before,
the reserve's value \textit{relative} to the now-lower market cap of the token has increased
to the reserve ratio.  This option is the ``pay now'' option.

Q:  Where's the ``don't pay'' option?

A:  You have to come up with some answer to where the ``backing'' for newly emitted
tokens will come from.  Unless there's no emission.  Or unless there's no ``backing'' for
any tokens.  So the ``don't pay'' option would be to have an SMT with either no
emission, or no market maker.

Q:  Don't fractional exponents require floating point to implement?

A:  Only if you need fairly high precision (we don't), don't care about bit-for-bit
reproducibility across compilers, OS's, CPU's, etc (we do), and need to do massive
numbers of calculations quickly (we don't).  A fast, approximate, all-integer
implementation is possible.

Q:  Does this market maker interact with the order book through the existing limit order
system, or is it a separate set of operations?

A:  In theory, it could be implemented either way.  The Steemit team is still deciding
which way to go.  In practice, even if implemented as a completely separate subsystem,
people will likely run arbitrage bots which will trade away any price differences
between the reserve system and the existing market system.

Q:  Where do the market maker's initial token balances come from?

A:  ICO units can specify the market maker as a destination.  An ICO creator
may direct a percentage of ICO's STEEM contributions to the MM by specifying
the market maker similarly to specifying a founder.  Or may use the soft cap
system to specify all STEEM above a pre-determined amount goes to the ICO.  Likewise,
a fixed or percentage amount of tokens can be added in the ICO to increase the MM's
token balance.

Q:  Can someone send STEEM or tokens to the market maker?

A:  Yes.

Q:  What are the side effects of sending STEEM or tokens to the market maker?

A:  The constants of integration are re-initialized, meaning the equilibrium price will
change.  The market maker will become more aggressive about selling the asset.

Q:  Can't this cause manipulation or appropriating the market maker's inventory
to private profit?

A:  Sending assets to the market maker does cause it to engage in trading activity
which affects the price.  However, dumping an identical amount on the market will
result in a larger amount of trading activity and a larger effect on the price.  If
Eve is willing to spend her tokens/STEEM to manipulate prices, she would prefer
the strategy of simply dumping tokens/STEEM on the market, as that strategy is more
cost-effective for her.

Q:  Does the market maker's activity generate profits (losses)?

A:  It depends on how you measure ``profits.'' If you measure the value of STEEM
and tokens in some external third currency such as US dollars or bitcoins, the
market maker's inventory, valued in that currency, can definitely increase or
decrease.  If people voluntarily send STEEM or tokens to the market maker,
such activity definitely increases the value of the market maker regardless of
your measurement.

Another way to define profits is by the constants of integration.  If both
of the constants of integration increase, or one increases while the other
remains the same, a tiny increase occurs with each trade when the market
maker is in ``taker'' mode.

Q:  What is ``taker'' mode?  How can a market maker be set to operate
in ``taker'' mode?

A:  When orders execute, the order used to set the price is called the maker;
the maker's counterparty is the taker.  In the STEEM on-chain market (and on
almost all trading platforms) the older order is always the maker.

When the market maker is in taker mode, its actions are always considered to be
taker orders, which execute at the price specified by the user acting as its
counterparty -- this price is always at least a little bit more favorable than
the market maker is willing to accept.  When the market maker is not in taker
mode, its actions are always considered to be maker orders, which don't
generate changes in the constants of integration.

Taker mode is a runtime parameter that can be set by the SMT's control account.

Q:  Who benefits from the profits of a market maker in taker mode?

A:  Maybe nobody, or maybe everybody.  It's decentralized.

Q:  OK, if my SMT reaches a steady price, the STEEM in the reserve is basically
locked up forever.  That seems not cool.  How do I set it up so that this STEEM
can be unlocked for the benefits of my SMT users?

A:  Set the DRR (decaying reserve ratio) setup parameter.  If you set DRR, then the
reserve ratio will slowly drop over time to a pre-set value, using its excess STEEM
reserves to buy excess tokens.  Setting DRR is an excellent, fair, decentralized
way to return excess capitalization to contributors in a more-popular-than-anticipated
ICO that raises more than the sponsor can effectively spend.

Q:  If the reserve ratio can change over time due to pay-later emissions or DRR,
it's not really a constant reserve ratio, is it?

A:  No, they're not.  The reserve ratio's called ``constant'' because it's constant
over the short-term, in normal conditions, or in the conditions in Bancor which is
where it was named.  But the name could be regarded as slightly misleading.

Q:  If the constants of integration can change over time, they're not really
constants either, are they?

A:  No, they're not.  They're called constants of integration because that's their
mathematical role in the calculation that introduces them.  Maybe they'll be
differently named in a future version of this paper.

Q:  Can I specify a contribution to a DRR to be a pay-later contribution, that
\textit{increases} its reserve ratio, the increase to be eventually negated over
time by future decay?  Why would I want to?

A:  Yes.  This is effectively contributing to the market maker,
subject to the condition that it's not allowed to immediately dump a portion
of the contribution.  It's useful if you want to make a large contribution
to a market maker without causing it to create a disturbance by immediately
dumping a significant fraction of your contribution onto the market.

Q:  Can I specify a DRR with emission to use pay-later for emissions when the
RR is decaying?

A:  Yes.

Q:  Is the market maker specified here equivalent to a Bancor token changer?

A:  No.  A Bancor token changer has multiple reserve ratios that must sum to
one hundred percent, and involves a third token that effectively represents
equity in the token changer.  This paper's market maker has none of these
features.

Q:  I want to have an initial ``price discovery'' period where people trade
without action from the market maker, then have tokens and STEEM from
the ICO gradually flow in over time to the market maker so it has a delayed,
slow start from zero to full power.  Can I do it?

A:  This is called ``gradual seeding'' and it may be supported.

Q:  What about numerical stability?

A:  A market maker will be restricted to only operate when its
balances exceed a certain minimum for both assets.  Also, reserve
ratios will be restricted to a certain range, all the mechanisms
that can set / increase / decrease a reserve ratio will be restricted
to not allow it to move outside the range.  Tentative numerical
experiments suggests these limits should be about 10,000 satoshis
of both assets, 5 percent and 50 percent, respectively.  These
values are subject to change based on future experimentation,
worst-case analysis, and testing.

\end{document}
